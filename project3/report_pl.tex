\documentclass[12pt]{article}
\usepackage[a4paper, margin=2.00cm]{geometry}
\usepackage[utf8]{inputenc}
\usepackage[polish]{babel}
\usepackage[T1]{fontenc}
\usepackage{graphicx}
\usepackage{amsmath}
\usepackage{minted}
\usepackage{parskip}
\usepackage{placeins}
\usepackage{enumitem}
\usepackage[justification=centering]{caption}
\usepackage{longtable}

\graphicspath{ {images/} }
\renewcommand{\familydefault}{\sfdefault}
\renewcommand{\thesection}{\arabic{section}.}
\renewcommand{\thesubsection}{\thesection\arabic{subsection}.}
\setminted{breaklines}
\hyphenpenalty=10000
\emergencystretch=\maxdimen
\captionsetup[table]{name=Tabela}
\captionsetup[figure]{name=Wykres}

\title{Sprawozdanie z wykonania zadania 3.14}
\author{Dawid Sygocki}
\date{\today}

\begin{document}
\makeatletter
{\huge \@title} \\
\textsl{\large Autor: \@author, nr indeksu: 304108, grupa: wtorek 14:15}
\makeatother

\section{Wprowadzenie}
Niniejszy raport przedstawia rezultaty wykonania zadanie projektowego nr 3 powiązanego z~szukaniem zer funkcji nieliniowych. Projekt dzielił się na dwa podzadania, z których pierwsze skupiało się na rzeczywistych miejscach zerowych prostej funkcji na wskazanym przedziale, a drugie na pierwiastkach rzeczywistych i zespolonych wielomianu.

\section{Zadanie I}
Przedmiotem badań była w tym zadaniu funkcja:
\[ f(x) = 3,1 - 3x - e^{-x}, \quad x \in \langle -5; 10 \rangle \]
Należało dokonać porównania dwóch zadanych metod: \textit{regula falsi} i Newtona.

Podsumowanie danych wejściowych i wyników stanowi wykres \ref{fig:zad1mix} przedstawiający przebieg funkcji, jej miejsca zerowe oraz startowe punkty/przedziały używanych algorytmów.

\begin{figure}[!htbp]
\centering
\includegraphics[width=16cm]{zad1mix}
\centering
\caption{Analiza badanego przedziału z zaznaczeniem wejścia badanych metod}
\label{fig:zad1mix}
\end{figure}
\FloatBarrier

\subsection{Metoda \textit{regula falsi}}
\textit{Regula falsi} (a. \textit{false position}) jest metodą globalnie zbieżną, ponieważ w miarę przesuwania granic przedziału zachowuje przedział izolacji pierwiastka. Jej zasada działania jest prosta -- dla granic przedziału leżących po różnych stronach zera wyznaczana jest sieczna przechodząca przez oś OX. Następnie pozycja jednego z krańców przedziału izolacji przesuwana jest do wyznaczonej wartości. Operacja jest powtarzana do momentu spełnienia wymagań dokładności wartości funkcji i/lub szerokości przedziału.

Alternatywą do tej metody, jeśli chodzi o globalną zbieżność, jest metoda bisekcji, gdzie zamiast prowadzenia siecznych przesuwa się jeden z krańców przedziału do jego środka. Wnioskować można, że szybkość zbieżności metody bisekcji jest niezależna od badanej funkcji, czego nie można powiedzieć o metodzie fałszywej reguły. W klasycznym wariancie jeden z krańców jest ,,przymocowany'', a~porusza się wyłącznie drugi, przez co zbieżność może być powolna, szczególnie dla wartości funkcji bliskich zeru.

Rozwiązaniem tego problemu jest modyfikacja ,,popychająca'' ku zeru krańce osiadłe w miejscu na dłuższy czas. Przy zachowaniu globalnego charakteru zbieżności metody pozwala to na poprawę jej charakterystyki -- z liniowej na superliniową.

Zgodnie z przedstawionym wcześniej rysunkiem \ref{fig:zad1mix}, podawane algorytmowi przedziały startowe były możliwie jak największe, sięgając od granicy ustalonego w zadaniu zakresu aż do drugiego miejsca zerowego. Sytuacja taka nie sprawiła metodzie żadnych problemów, z właściwą sobie szybkością zbiegała ona do celu.

Ze względu na pokaźną objętość, dokładne tabele z wartościami w każdej iteracji umieszczone zostały na końcu dokumentu, w sekcji \ref{sec:samples} (po wydrukach kodu).  
Warto jednak już teraz wspomnieć, że~przy dokładności \( 10^{-12} \) algorytm potrzebował dla zakresu \( \langle -5; 0.897 \rangle \) aż 261 kroków, natomiast dla \( \langle -2.303; 10 \rangle \) 20 kroków.  
Porównanie charakterystyki zbieżności zostanie przedstawione w sekcji dotyczącej metody stycznych.

\subsection{Metoda Newtona (stycznych)}

Metoda Newtona, w odróżnieniu od poprzednio badanej, jest zbieżna kwadratowo, ale jedynie lokalnie. Oznacza to, że poza obszarem atrakcji pierwiastka może być rozbieżna. Dobrze więc wpierw doprecyzować zakres poszukiwań metodą o działaniu globalnym. Kolejną różnicą jest, że jako wejściowy argument metoda stycznych przyjmuje pojedynczy punkt, nie zaś dwie granice przedziału.

Algorytm wykorzystuje w każdym kroku liniowe przybliżenie funkcji (za pomocą pochodnej), zbliżając się w ten sposób do rozwiązania tym szybciej, im bardziej ,,stroma'' jest funkcja w otoczeniu pierwiastka.  
Jeśli jednak funkcja w pobliżu zera jest wypłaszczona, metoda przyniesie słabe efekty (szczególnie dla pochodnej bliskiej zeru).

Na cele zadania pochodna badanej funkcji została wyznaczona ręcznie:
\[ y = e^{-x} - 3 \]

Jak poprzednio, oczekiwana dokładność wynosiła \( 10^{-12} \).

Na początek podsumowanie liczbowe: dla punktu startowego \( x = -5 \) algorytm musiał wykonać 8~kroków, a dla \( x = 10 \) zaledwie 4.

Teraz pora na wspomniane wcześniej porównanie charakterystyki zbieżności. Przedstawione jest ono na poniższych wykresach (\ref{fig:zad1comp1} i \ref{fig:zad1comp2}). Jako pierwsze miejsce zerowe została przyjęta wartość ok. 0,8973, a jako drugie ok. -2,3037.

\begin{figure}[!htbp]
\centering
\includegraphics[width=16cm]{zad1comp1}
\centering
\caption{Porównanie zbieżności metod w okolicach pierwszego miejsca zerowego}
\label{fig:zad1comp1}
\end{figure}
\FloatBarrier

\begin{figure}[!htbp]
\centering
\includegraphics[width=16cm]{zad1comp2}
\centering
\caption{Porównanie zbieżności metod w okolicach drugiego miejsca zerowego}
\label{fig:zad1comp2}
\end{figure}
\FloatBarrier

Na wykresach widać wyraźnie, że metoda Newtona cechuje się o wiele szybszą zbieżnością. Kilkoma sporymi skokami zmierza do rozwiązania, a potem je ostatecznie poprawia. Warto zaznaczyć, że~punkty startowe są dla metody wygodne, ponieważ funkcja jest stroma (szczególnie na obrzeżach).

Metoda fałszywej reguły radzi sobie gorzej. Na początku i pod koniec działania algorytmu pozycja przybliżenie zmienia się bardzo powoli. Początkowy wolny start może wynikać z tego, że, mając do wyboru dwa krańce, algorytm wybrał ten sporo mniej stromy (doświadczalnie okazało się, że~implementacja przesuwa w obu przypadkach właśnie kraniec leżący ponad osią OX). Spowolnienie końcowe jest rezultatem zastosowania najprostszej, niezmodyfikowanej implementacji, w której zakres z jednej strony zbiega do pierwiastka, a z drugiej pozostaje w miejscu, przez co dłużej zajmuje spełnienie wymagań dot. minimalnej szerokości przedziału oraz dokładności zera.

Jednakże, metoda Newtona nie jest idealna. Ciekawym odkryciem było, że podanie jej jako punktu startowego \( x = -1,1 \) skutkuje rozbieżnością. Takiego efektu bardziej spodziewałem się po dokładnym wierzchołku funkcji znajdującym się w punkcie \( x = -1,0986 \) (obliczonym ręcznie na~podstawie przyrównania pochodnej do zera, a więc uczynienia ją równoległą do osi OX, co tworzy sytuację blokującą algebraicznie -- dzielenie przez zero). Tam jednak metoda Newtona podaje przybliżenie poprawne, wykonując jedynie z początku potężny skok, a potem się stabilizując. Całkiem możliwe, że jest to skutek niedokładności maszynowej.

\section{Zadanie II}

Drugie zadanie traktowało o szukaniu pierwiastków rzeczywistych i zespolonych podanego wielomianu:
\[ f(x) = -x^4 - 7x^3 + 7x^2 + 3x + 9 \]

Wykorzystanym algorytmem miała być metoda Müllera, konkretnie wersja identyfikowana w~podręczniku jako MM2, wykorzystująca wartość funkcji i jej dwóch pierwszych pochodnych w~tym samym punkcie do poznania współczynników \(a\), \(b\) i \(c\) kolejnego przybliżenia. Wariant ten jest lekko efektywniejszy obliczeniowo niż interpolacja funkcją kwadratową w trzech punktach. Wielokrotna różniczkowalność właściwa wielomianom zapewnia, że pochodna drugiego rzędu zawsze będzie istniała.

Metoda Müllera jest zbieżna lokalnie i prawie kwadratowo (rząd zbieżności = 1,84), a więc ma nieznacznie gorsze parametry od metody Newtona.

W procesie obliczania kolejnych pierwiastków ważna jest deflacja, która pozwala ,,oczyścić'' wielomian z już odnalezionych zer. Dzięki temu możliwe jest rozpoczynanie szukania w tym samym wybranym miejscu bez obaw, że niewłaściwie wskaże się ponownie ten sam pierwiastek. Można też przy otrzymaniu odpowiednio niskiego stopnia wielomianu przejść do rozwiązywania go prostszą metodą (np. jako funkcji kwadratowej).  
W swojej implementacji wykorzystuję do deflacji prosty schemat Hornera. Nie użyłem wersji sklejanej z uwagi na niewielką liczbę współczynników. Postanowiłem jednak dla pewności zastosować \textit{root polishing}, czyli doprecyzowanie wartości pierwiastków wyznaczonych na podstawie wielomianu będącego efektem dzielenia, korzystając z~oryginalnego wielomianu (tj. poprzez podanie przybliżonego pierwiastka jako punktu startowego dla np. bardziej wydajnej metody Newtona). Okazuje się, że dla danych z~zadania i to jest przesadną dokładnością, ponieważ nie doświadczyłem zysku precyzji (dla~określonej dokładności \( 10^{-12} \)).

W swoim rozwiązaniu zadania drugiego używam zera jako wartości startowej ze względu na znalezioną w podręczniku do przedmiotu poradę głoszącą, że wyznaczanie pierwiastków od tych z~najmniejszym modułem do tych z największym prowadzi do zmniejszenia błędów obliczeniowych przy deflacji.

Dla kolejnych wielomianów (coraz niższego stopnia) algorytm znajduje miejsca zerowe po 4, 3 i~wreszcie dwa razy po jednym kroku (dla trywialnego przypadku funkcji kwadratowej i liniowej).

\begin{table}[!htbp]
\centering
\begin{longtable}{|c|}
    \hline
    -0.3273 + 0.8054i \\ \hline
    -0.3273 - 0.8054i \\ \hline
    1.5151 \\ \hline
    -7.8605 \\ \hline
\end{longtable}
\label{table:roots}
\caption{Lista pierwiastków badanej funkcji wyliczona za pomocą metody Müllera MM2}
\end{table}
\FloatBarrier

\clearpage
\section{Wydruk wykorzystanego kodu}

\subsection{regulafalsi.m}
Implementacja metody \textit{regula falsi} dla zadania I.
\begin{minted}{matlab}
function root = regulafalsi(func, startRange, valueTolerance, rangeWidthTolerance)
    assert(size(startRange, 2) == 2, ...
        "startRange: Expected vector of size 2");
    assert(func(startRange(1)) * func(startRange(2)) < 0, ...
        "func, startRange: Expected function values to be opposite at the ends of specified range");
    assert(valueTolerance > 0);
    assert(rangeWidthTolerance > 0);

    a = min(startRange);
    b = max(startRange);

    while min(abs(func(a)), abs(func(b))) > valueTolerance && b - a > rangeWidthTolerance
        c = (a * func(b) - b * func(a)) / (func(b) - func(a));
        if func(a) * func(c) < 0
            b = c;
        else
            a = c;
        end
    end

    if abs(func(a)) < abs(func(b))
        root = a;
    else
        root = b;
    end
end
\end{minted}

\subsection{newton.m}
Implementacja metody Newtona dla zadania I.

W przypadku rozbieżności szybko dochodzi do dzielenia wartości bliskich nieskończoności, co generuje w wyniku wartość NaN.
\begin{minted}{matlab}
function root = newton(func, derivative, startPoint, valueTolerance)
    assert(valueTolerance > 0);

    root = startPoint;

    while abs(func(root)) > valueTolerance
        root = root - func(root) / derivative(root)
    end
end

\end{minted}

\subsection{exponentialf.m}
Funkcja wykorzystana w zadaniu I.
\begin{minted}{matlab}
function y = exponentialf(x)
    y = -exp(-x) - 3*x + 3.1;
end
\end{minted}

\subsection{exponentiald1.m}
Pochodna funkcja w zadania I, potrzebna przy korzystaniu z metody Newtona.
\begin{minted}{matlab}
function y = exponentiald1(x)
    y = exp(-x) - 3;
end
\end{minted}

\subsection{task2.m}
Funkcja realizująca zadanie II.
\begin{minted}{matlab}
function roots = task2(valueTolerance)
    assert(valueTolerance > 0);

    a = [-1, -7, 7, 3, 9];
    roots = zeros(1, length(a) - 1);

    q = a;
    for i=1:length(roots)
        roots(i) = mullerpolynomial(q, 0, valueTolerance);
        % root polishing using Newton method
        if i > 1
            roots(i) = newton(@(x) polynomialf(x, q), @(x) polynomiald1(x, q), roots(i), valueTolerance);
        end
        q = horner(q, roots(i));
    end
end
\end{minted}

\subsection{mullerpolynomial.m}
Implementacja algorytmu Müllera z użyciem pochodnych (MM2) służąca do szukania pierwiastków wielomianu z zadania II.
\begin{minted}{matlab}
function root = mullerpolynomial(coeffs, startPoint, valueTolerance)
    % coeffs should be ordered from greatest to least index
    % e.g. [a4, a3, a2, a1, a0]
    assert(valueTolerance > 0);

    root = startPoint;
    while abs(polynomialf(root, coeffs)) > valueTolerance
        funcValue = polynomialf(root, coeffs);
        deriv1Value = polynomiald1(root, coeffs);
        deriv2Value = polynomiald2(root, coeffs);
        % divisor for z+
        pDivisor = deriv1Value + sqrt(deriv1Value^2 - 2*funcValue*deriv2Value);
        % divisor for z-
        nDivisor = deriv1Value - sqrt(deriv1Value^2 - 2*funcValue*deriv2Value);
        if abs(pDivisor) >= abs(nDivisor)
            zMin = -2*funcValue / pDivisor;
        else
            zMin = -2*funcValue / nDivisor;
        end
        root = root + zMin;
    end
end
\end{minted}

\subsection{horner.m}
Prosta implementacja deflacji czynnikiem liniowym - dzielenia wielomianów schematem Hornera.
\begin{minted}{matlab}
function q = horner(a, root)
    % coeffs should be ordered from greatest to least index
    % e.g. [a4, a3, a2, a1, a0]
    q = zeros(1, length(a));
    for i=2:length(q)
        q(i) = a(i - 1) + q(i - 1) * root;
    end
    q = q(2:end);
end
\end{minted}

\subsection{polynomialf.m}
Funkcja wielomianowa parametryzowana współczynnikami.
\begin{minted}{matlab}
function y = polynomialf(x, coeffs)
    % coeffs should be ordered from greatest to least index
    % e.g. [a4, a3, a2, a1, a0]
    y = sum(x .^ flip(0:(length(coeffs) - 1)) ...
        .* coeffs);
end
\end{minted}

\subsection{polynomiald1.m}
Pierwsza pochodna funkcji wielomianowej parametryzowana jej współczynnikami.
\begin{minted}{matlab}
function y = polynomiald1(x, coeffs)
    % coeffs should be ordered from greatest to least index
    % e.g. [a4, a3, a2, a1, a0]
    y = sum(x .^ flip(0:(length(coeffs) - 2)) ...
        .* flip(1:(length(coeffs) - 1)) ...
        .* coeffs(1:(end - 1)));
end
\end{minted}

\subsection{polynomiald2.m}
Druga pochodna funkcji wielomianowej parametryzowana jej współczynnikami.
\begin{minted}{matlab}
function y = polynomiald2(x, coeffs)
    % coeffs should be ordered from greatest to least index
    % e.g. [a4, a3, a2, a1, a0]
    y = sum(x .^ flip(0:(length(coeffs) - 3)) ...
        .* flip(1:(length(coeffs) - 2)) ...
        .* flip(2:(length(coeffs) - 1)) ...
        .* coeffs(1:(end - 2)));
end
\end{minted}

\clearpage
\section{Punkty otrzymane w kolejnych iteracjach algorytmów wykorzystanych w zadaniu I}
\label{sec:samples}

\subsection{Metoda \textit{regula falsi}, pierwsze miejsce zerowe}
\begin{center}
\begin{longtable}{|c|c|c|} 
    \hline
    \textbf{Iteracja} & \textbf{x} & \textbf{f(x)} \\
    \hline\hline
    0	&     8,970E-01	&   1,209E-03 \\ \hline
    1	&     8,969E-01	&   1,351E-03 \\ \hline
    2	&     8,969E-01	&   1,509E-03 \\ \hline
    3	&     8,968E-01	&   1,686E-03 \\ \hline
    4	&     8,967E-01	&   1,884E-03 \\ \hline
    5	&     8,967E-01	&   2,105E-03 \\ \hline
    6	&     8,966E-01	&   2,352E-03 \\ \hline
    7	&     8,965E-01	&   2,627E-03 \\ \hline
    8	&     8,963E-01	&   2,936E-03 \\ \hline
    9	&     8,962E-01	&   3,280E-03 \\ \hline
    10	&     8,961E-01	&   3,664E-03 \\ \hline
    11	&     8,959E-01	&   4,094E-03 \\ \hline
    12	&     8,957E-01	&   4,574E-03 \\ \hline
    13	&     8,955E-01	&   5,111E-03 \\ \hline
    14	&     8,953E-01	&   5,710E-03 \\ \hline
    15	&     8,950E-01	&   6,379E-03 \\ \hline
    16	&     8,947E-01	&   7,127E-03 \\ \hline
    17	&     8,944E-01	&   7,962E-03 \\ \hline
    18	&     8,940E-01	&   8,895E-03 \\ \hline
    19	&     8,936E-01	&   9,937E-03 \\ \hline
    20	&     8,932E-01	&   1,110E-02 \\ \hline
    21	&     8,927E-01	&   1,240E-02 \\ \hline
    22	&     8,921E-01	&   1,385E-02 \\ \hline
    23	&     8,915E-01	&   1,548E-02 \\ \hline
    24	&     8,908E-01	&   1,729E-02 \\ \hline
    25	&     8,900E-01	&   1,931E-02 \\ \hline
    26	&     8,891E-01	&   2,157E-02 \\ \hline
    27	&     8,882E-01	&   2,410E-02 \\ \hline
    28	&     8,871E-01	&   2,691E-02 \\ \hline
    29	&     8,859E-01	&   3,006E-02 \\ \hline
    30	&     8,845E-01	&   3,357E-02 \\ \hline
    31	&     8,830E-01	&   3,749E-02 \\ \hline
    32	&     8,813E-01	&   4,187E-02 \\ \hline
    33	&     8,794E-01	&   4,675E-02 \\ \hline
    34	&     8,773E-01	&   5,220E-02 \\ \hline
    35	&     8,749E-01	&   5,828E-02 \\ \hline
    36	&     8,723E-01	&   6,506E-02 \\ \hline
    37	&     8,694E-01	&   7,263E-02 \\ \hline
    38	&     8,661E-01	&   8,106E-02 \\ \hline
    39	&     8,625E-01	&   9,047E-02 \\ \hline
    40	&     8,584E-01	&   1,009E-01 \\ \hline
    41	&     8,539E-01	&   1,126E-01 \\ \hline
    42	&     8,488E-01	&   1,256E-01 \\ \hline
    43	&     8,432E-01	&   1,401E-01 \\ \hline
    44	&     8,369E-01	&   1,562E-01 \\ \hline
    45	&     8,299E-01	&   1,742E-01 \\ \hline
    46	&     8,221E-01	&   1,941E-01 \\ \hline
    47	&     8,135E-01	&   2,163E-01 \\ \hline
    48	&     8,038E-01	&   2,409E-01 \\ \hline
    49	&     7,931E-01	&   2,682E-01 \\ \hline
    50	&     7,812E-01	&   2,984E-01 \\ \hline
    51	&     7,680E-01	&   3,320E-01 \\ \hline
    52	&     7,534E-01	&   3,691E-01 \\ \hline
    53	&     7,371E-01	&   4,102E-01 \\ \hline
    54	&     7,191E-01	&   4,555E-01 \\ \hline
    55	&     6,992E-01	&   5,054E-01 \\ \hline
    56	&     6,772E-01	&   5,604E-01 \\ \hline
    57	&     6,529E-01	&   6,208E-01 \\ \hline
    58	&     6,261E-01	&   6,871E-01 \\ \hline
    59	&     5,966E-01	&   7,596E-01 \\ \hline
    60	&     5,641E-01	&   8,388E-01 \\ \hline
    61	&     5,285E-01	&   9,249E-01 \\ \hline
    62	&     4,896E-01	&   1,018E+00 \\ \hline
    63	&     4,470E-01	&   1,119E+00 \\ \hline
    64	&     4,006E-01	&   1,228E+00 \\ \hline
    65	&     3,502E-01	&   1,345E+00 \\ \hline
    66	&     2,955E-01	&   1,469E+00 \\ \hline
    67	&     2,365E-01	&   1,601E+00 \\ \hline
    68	&     1,729E-01	&   1,740E+00 \\ \hline
    69	&     1,048E-01	&   1,885E+00 \\ \hline
    70	&     3,198E-02	&   2,036E+00 \\ \hline
    71	&     -4,541E-02	&   2,190E+00 \\ \hline
    72	&     -1,273E-01	&   2,346E+00 \\ \hline
    73	&     -2,135E-01	&   2,502E+00 \\ \hline
    74	&     -3,037E-01	&   2,656E+00 \\ \hline
    75	&     -3,975E-01	&   2,804E+00 \\ \hline
    76	&     -4,944E-01	&   2,944E+00 \\ \hline
    77	&     -5,940E-01	&   3,071E+00 \\ \hline
    78	&     -6,954E-01	&   3,182E+00 \\ \hline
    79	&     -7,980E-01	&   3,273E+00 \\ \hline
    80	&     -9,009E-01	&   3,341E+00 \\ \hline
    81	&     -1,003E+00	&   3,383E+00 \\ \hline
    82	&     -1,105E+00	&   3,396E+00 \\ \hline
    83	&     -1,203E+00	&   3,379E+00 \\ \hline
    84	&     -1,299E+00	&   3,331E+00 \\ \hline
    85	&     -1,392E+00	&   3,253E+00 \\ \hline
    86	&     -1,480E+00	&   3,148E+00 \\ \hline
    87	&     -1,563E+00	&   3,017E+00 \\ \hline
    88	&     -1,640E+00	&   2,864E+00 \\ \hline
    89	&     -1,713E+00	&   2,694E+00 \\ \hline
    90	&     -1,779E+00	&   2,513E+00 \\ \hline
    91	&     -1,840E+00	&   2,323E+00 \\ \hline
    92	&     -1,895E+00	&   2,131E+00 \\ \hline
    93	&     -1,945E+00	&   1,940E+00 \\ \hline
    94	&     -1,990E+00	&   1,754E+00 \\ \hline
    95	&     -2,030E+00	&   1,575E+00 \\ \hline
    96	&     -2,066E+00	&   1,407E+00 \\ \hline
    97	&     -2,097E+00	&   1,249E+00 \\ \hline
    98	&     -2,125E+00	&   1,105E+00 \\ \hline
    99	&     -2,149E+00	&   9,723E-01 \\ \hline
    100	&     -2,170E+00	&   8,527E-01 \\ \hline
    101	&     -2,188E+00	&   7,453E-01 \\ \hline
    102	&     -2,204E+00	&   6,495E-01 \\ \hline
    103	&     -2,218E+00	&   5,646E-01 \\ \hline
    104	&     -2,230E+00	&   4,896E-01 \\ \hline
    105	&     -2,240E+00	&   4,238E-01 \\ \hline
    106	&     -2,249E+00	&   3,662E-01 \\ \hline
    107	&     -2,257E+00	&   3,159E-01 \\ \hline
    108	&     -2,264E+00	&   2,722E-01 \\ \hline
    109	&     -2,269E+00	&   2,343E-01 \\ \hline
    110	&     -2,274E+00	&   2,015E-01 \\ \hline
    111	&     -2,279E+00	&   1,731E-01 \\ \hline
    112	&     -2,282E+00	&   1,486E-01 \\ \hline
    113	&     -2,285E+00	&   1,275E-01 \\ \hline
    114	&     -2,288E+00	&   1,094E-01 \\ \hline
    115	&     -2,290E+00	&   9,377E-02 \\ \hline
    116	&     -2,292E+00	&   8,035E-02 \\ \hline
    117	&     -2,294E+00	&   6,883E-02 \\ \hline
    118	&     -2,295E+00	&   5,895E-02 \\ \hline
    119	&     -2,296E+00	&   5,047E-02 \\ \hline
    120	&     -2,298E+00	&   4,320E-02 \\ \hline
    121	&     -2,298E+00	&   3,697E-02 \\ \hline
    122	&     -2,299E+00	&   3,164E-02 \\ \hline
    123	&     -2,300E+00	&   2,707E-02 \\ \hline
    124	&     -2,300E+00	&   2,316E-02 \\ \hline
    125	&     -2,301E+00	&   1,981E-02 \\ \hline
    126	&     -2,301E+00	&   1,694E-02 \\ \hline
    127	&     -2,302E+00	&   1,449E-02 \\ \hline
    128	&     -2,302E+00	&   1,239E-02 \\ \hline
    129	&     -2,302E+00	&   1,060E-02 \\ \hline
    130	&     -2,302E+00	&   9,064E-03 \\ \hline
    131	&     -2,303E+00	&   7,751E-03 \\ \hline
    132	&     -2,303E+00	&   6,628E-03 \\ \hline
    133	&     -2,303E+00	&   5,667E-03 \\ \hline
    134	&     -2,303E+00	&   4,846E-03 \\ \hline
    135	&     -2,303E+00	&   4,143E-03 \\ \hline
    136	&     -2,303E+00	&   3,542E-03 \\ \hline
    137	&     -2,303E+00	&   3,029E-03 \\ \hline
    138	&     -2,303E+00	&   2,590E-03 \\ \hline
    139	&     -2,303E+00	&   2,214E-03 \\ \hline
    140	&     -2,303E+00	&   1,893E-03 \\ \hline
    141	&     -2,303E+00	&   1,618E-03 \\ \hline
    142	&     -2,303E+00	&   1,384E-03 \\ \hline
    143	&     -2,304E+00	&   1,183E-03 \\ \hline
    144	&     -2,304E+00	&   1,011E-03 \\ \hline
    145	&     -2,304E+00	&   8,648E-04 \\ \hline
    146	&     -2,304E+00	&   7,393E-04 \\ \hline
    147	&     -2,304E+00	&   6,321E-04 \\ \hline
    148	&     -2,304E+00	&   5,404E-04 \\ \hline
    149	&     -2,304E+00	&   4,620E-04 \\ \hline
    150	&     -2,304E+00	&   3,950E-04 \\ \hline
    151	&     -2,304E+00	&   3,377E-04 \\ \hline
    152	&     -2,304E+00	&   2,887E-04 \\ \hline
    153	&     -2,304E+00	&   2,468E-04 \\ \hline
    154	&     -2,304E+00	&   2,110E-04 \\ \hline
    155	&     -2,304E+00	&   1,804E-04 \\ \hline
    156	&     -2,304E+00	&   1,542E-04 \\ \hline
    157	&     -2,304E+00	&   1,319E-04 \\ \hline
    158	&     -2,304E+00	&   1,127E-04 \\ \hline
    159	&     -2,304E+00	&   9,638E-05 \\ \hline
    160	&     -2,304E+00	&   8,240E-05 \\ \hline
    161	&     -2,304E+00	&   7,045E-05 \\ \hline
    162	&     -2,304E+00	&   6,023E-05 \\ \hline
    163	&     -2,304E+00	&   5,149E-05 \\ \hline
    164	&     -2,304E+00	&   4,402E-05 \\ \hline
    165	&     -2,304E+00	&   3,764E-05 \\ \hline
    166	&     -2,304E+00	&   3,218E-05 \\ \hline
    167	&     -2,304E+00	&   2,751E-05 \\ \hline
    168	&     -2,304E+00	&   2,352E-05 \\ \hline
    169	&     -2,304E+00	&   2,011E-05 \\ \hline
    170	&     -2,304E+00	&   1,719E-05 \\ \hline
    171	&     -2,304E+00	&   1,470E-05 \\ \hline
    172	&     -2,304E+00	&   1,256E-05 \\ \hline
    173	&     -2,304E+00	&   1,074E-05 \\ \hline
    174	&     -2,304E+00	&   9,183E-06 \\ \hline
    175	&     -2,304E+00	&   7,851E-06 \\ \hline
    176	&     -2,304E+00	&   6,712E-06 \\ \hline
    177	&     -2,304E+00	&   5,739E-06 \\ \hline
    178	&     -2,304E+00	&   4,906E-06 \\ \hline
    179	&     -2,304E+00	&   4,194E-06 \\ \hline
    180	&     -2,304E+00	&   3,586E-06 \\ \hline
    181	&     -2,304E+00	&   3,066E-06 \\ \hline
    182	&     -2,304E+00	&   2,621E-06 \\ \hline
    183	&     -2,304E+00	&   2,241E-06 \\ \hline
    184	&     -2,304E+00	&   1,916E-06 \\ \hline
    185	&     -2,304E+00	&   1,638E-06 \\ \hline
    186	&     -2,304E+00	&   1,400E-06 \\ \hline
    187	&     -2,304E+00	&   1,197E-06 \\ \hline
    188	&     -2,304E+00	&   1,023E-06 \\ \hline
    189	&     -2,304E+00	&   8,750E-07 \\ \hline
    190	&     -2,304E+00	&   7,480E-07 \\ \hline
    191	&     -2,304E+00	&   6,395E-07 \\ \hline
    192	&     -2,304E+00	&   5,468E-07 \\ \hline
    193	&     -2,304E+00	&   4,674E-07 \\ \hline
    194	&     -2,304E+00	&   3,996E-07 \\ \hline
    195	&     -2,304E+00	&   3,417E-07 \\ \hline
    196	&     -2,304E+00	&   2,921E-07 \\ \hline
    197	&     -2,304E+00	&   2,497E-07 \\ \hline
    198	&     -2,304E+00	&   2,135E-07 \\ \hline
    199	&     -2,304E+00	&   1,825E-07 \\ \hline
    200	&     -2,304E+00	&   1,560E-07 \\ \hline
    201	&     -2,304E+00	&   1,334E-07 \\ \hline
    202	&     -2,304E+00	&   1,141E-07 \\ \hline
    203	&     -2,304E+00	&   9,751E-08 \\ \hline
    204	&     -2,304E+00	&   8,336E-08 \\ \hline
    205	&     -2,304E+00	&   7,127E-08 \\ \hline
    206	&     -2,304E+00	&   6,093E-08 \\ \hline
    207	&     -2,304E+00	&   5,209E-08 \\ \hline
    208	&     -2,304E+00	&   4,454E-08 \\ \hline
    209	&     -2,304E+00	&   3,808E-08 \\ \hline
    210	&     -2,304E+00	&   3,255E-08 \\ \hline
    211	&     -2,304E+00	&   2,783E-08 \\ \hline
    212	&     -2,304E+00	&   2,379E-08 \\ \hline
    213	&     -2,304E+00	&   2,034E-08 \\ \hline
    214	&     -2,304E+00	&   1,739E-08 \\ \hline
    215	&     -2,304E+00	&   1,487E-08 \\ \hline
    216	&     -2,304E+00	&   1,271E-08 \\ \hline
    217	&     -2,304E+00	&   1,087E-08 \\ \hline
    218	&     -2,304E+00	&   9,290E-09 \\ \hline
    219	&     -2,304E+00	&   7,943E-09 \\ \hline
    220	&     -2,304E+00	&   6,791E-09 \\ \hline
    221	&     -2,304E+00	&   5,805E-09 \\ \hline
    222	&     -2,304E+00	&   4,963E-09 \\ \hline
    223	&     -2,304E+00	&   4,243E-09 \\ \hline
    224	&     -2,304E+00	&   3,628E-09 \\ \hline
    225	&     -2,304E+00	&   3,102E-09 \\ \hline
    226	&     -2,304E+00	&   2,652E-09 \\ \hline
    227	&     -2,304E+00	&   2,267E-09 \\ \hline
    228	&     -2,304E+00	&   1,938E-09 \\ \hline
    229	&     -2,304E+00	&   1,657E-09 \\ \hline
    230	&     -2,304E+00	&   1,417E-09 \\ \hline
    231	&     -2,304E+00	&   1,211E-09 \\ \hline
    232	&     -2,304E+00	&   1,035E-09 \\ \hline
    233	&     -2,304E+00	&   8,852E-10 \\ \hline
    234	&     -2,304E+00	&   7,568E-10 \\ \hline
    235	&     -2,304E+00	&   6,470E-10 \\ \hline
    236	&     -2,304E+00	&   5,532E-10 \\ \hline
    237	&     -2,304E+00	&   4,729E-10 \\ \hline
    238	&     -2,304E+00	&   4,043E-10 \\ \hline
    239	&     -2,304E+00	&   3,457E-10 \\ \hline
    240	&     -2,304E+00	&   2,955E-10 \\ \hline
    241	&     -2,304E+00	&   2,527E-10 \\ \hline
    242	&     -2,304E+00	&   2,160E-10 \\ \hline
    243	&     -2,304E+00	&   1,847E-10 \\ \hline
    244	&     -2,304E+00	&   1,579E-10 \\ \hline
    245	&     -2,304E+00	&   1,350E-10 \\ \hline
    246	&     -2,304E+00	&   1,154E-10 \\ \hline
    247	&     -2,304E+00	&   9,866E-11 \\ \hline
    248	&     -2,304E+00	&   8,436E-11 \\ \hline
    249	&     -2,304E+00	&   7,216E-11 \\ \hline
    250	&     -2,304E+00	&   6,164E-11 \\ \hline
    251	&     -2,304E+00	&   5,274E-11 \\ \hline
    252	&     -2,304E+00	&   4,510E-11 \\ \hline
    253	&     -2,304E+00	&   3,858E-11 \\ \hline
    254	&     -2,304E+00	&   3,296E-11 \\ \hline
    255	&     -2,304E+00	&   2,820E-11 \\ \hline
    256	&     -2,304E+00	&   2,413E-11 \\ \hline
    257	&     -2,304E+00	&   2,062E-11 \\ \hline
    258	&     -2,304E+00	&   1,761E-11 \\ \hline
    259	&     -2,304E+00	&   1,509E-11 \\ \hline
    260	&     -2,304E+00	&   1,291E-11 \\ \hline
    261	&     -2,304E+00	&   1,102E-11 \\ \hline
    262	&     -2,304E+00	&   9,408E-12 \\ \hline
\end{longtable}
\label{table:samples1}
\end{center}
\FloatBarrier

\subsection{Metoda Newtona, pierwsze miejsce zerowe}
\begin{table}[!htbp]
\centering
\begin{longtable}{|c|c|c|} 
    \hline
    \textbf{Iteracja} & \textbf{x} & \textbf{f(x)} \\
    \hline\hline
    1 & -5,000E+00  &	-1,303E+02   \\ \hline
    2 & -4,104E+00  &	-4,516E+01   \\ \hline
    3 & -3,319E+00  &	-1,459E+01   \\ \hline
    4 & -2,728E+00  &	-4,013E+00   \\ \hline
    5 & -2,401E+00  &	-7,328E-01   \\ \hline
    6 & -2,310E+00  &	-4,451E-02   \\ \hline
    7 & -2,304E+00  &	-1,990E-04   \\ \hline
    8 & -2,304E+00  &	-4,033E-09   \\ \hline
    9 & -2,304E+00  &	1,377E-14   \\ \hline
\end{longtable}
\label{table:samples2}
\end{table}
\FloatBarrier

\clearpage
\subsection{Metoda \textit{regula falsi}, drugie miejsce zerowe}
\begin{table}[!htbp]
\centering
\begin{longtable}{|c|c|c|} 
    \hline
    \textbf{Iteracja} & \textbf{x} & \textbf{f(x)} \\
    \hline\hline
    1 & -2,303E+00  &	4,850E-03   \\ \hline
    2 & -2,301E+00  &	2,036E-02   \\ \hline
    3 & -2,291E+00  &	8,488E-02   \\ \hline
    4 & -2,253E+00  &	3,440E-01   \\ \hline
    5 & -2,098E+00  &	1,244E+00   \\ \hline
    6 & -1,564E+00  &	3,015E+00   \\ \hline
    7 & -3,982E-01  &	2,805E+00   \\ \hline
    8 & 5,838E-01   &	7,907E-01   \\ \hline
    9 & 8,527E-01   &	1,156E-01   \\ \hline
    10 & 8,919E-01   &	1,454E-02   \\ \hline
    11 & 8,968E-01   &	1,790E-03   \\ \hline
    12 & 8,974E-01   &	2,198E-04   \\ \hline
    13 & 8,975E-01   &	2,699E-05   \\ \hline
    14 & 8,975E-01   &	3,314E-06   \\ \hline
    15 & 8,975E-01   &	4,069E-07   \\ \hline
    16 & 8,975E-01   &	4,995E-08   \\ \hline
    17 & 8,975E-01   &	6,133E-09   \\ \hline
    18 & 8,975E-01   &	7,530E-10   \\ \hline
    19 & 8,975E-01   &	9,245E-11   \\ \hline
    20 & 8,975E-01   &	1,135E-11   \\ \hline
    21 & 8,975E-01   &	1,393E-12   \\ \hline
\end{longtable}
\label{table:samples3}
\end{table}
\FloatBarrier

\subsection{Metoda Newtona, drugie miejsce zerowe}
\begin{table}[!htbp]
\centering
\begin{longtable}{|c|c|c|} 
    \hline
    \textbf{Iteracja} & \textbf{x} & \textbf{f(x)} \\
    \hline\hline
    1 & 1,000E+01   &	-2,690E+01   \\ \hline
    2 & 1,033E+00   &	-3,554E-01   \\ \hline
    3 & 8,988E-01   &	-3,364E-03   \\ \hline
    4 & 8,975E-01   &	-3,427E-07   \\ \hline
    5 & 8,975E-01   &	-3,997E-15   \\ \hline
\end{longtable}
\label{table:samples4}
\end{table}
\FloatBarrier

\end{document}