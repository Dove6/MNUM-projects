\documentclass[12pt]{article}
\usepackage[a4paper, margin=2.00cm]{geometry}
\usepackage[utf8]{inputenc}
\usepackage[polish]{babel}
\usepackage[T1]{fontenc}
\usepackage{graphicx}
\usepackage{amsmath}
\usepackage{minted}
\usepackage{parskip}
\usepackage{placeins}
\usepackage{enumitem}
\usepackage[justification=centering]{caption}
\usepackage{longtable}
\usepackage{multirow}
\usepackage{multicol}

\graphicspath{ {images/} }
\renewcommand{\familydefault}{\sfdefault}
\renewcommand{\thesection}{\arabic{section}.}
\renewcommand{\thesubsection}{\thesection\arabic{subsection}.}
\setminted{breaklines}
\hyphenpenalty=10000
\emergencystretch=\maxdimen
\captionsetup[table]{name=Tabela}
\captionsetup[figure]{name=Wykres}

\title{Sprawozdanie z wykonania zadania 4.14}
\author{Dawid Sygocki}
\date{\today}

\begin{document}
\makeatletter
{\huge \@title} \\
\textsl{\large Autor: \@author, nr indeksu: 304108, grupa: wtorek 14:15}
\makeatother

\section{Wprowadzenie}
Niniejszy raport przedstawia rezultaty wykonania zadania projektowego nr 4, które skupiało się na metodach różnicowych. Projekt dzielił się na trzy zadania, gdzie każde odpowiadało implementacji jednej z metod: klasycznej Rungego-Kutty ze stałym krokiem, predyktora-korektora na podstawie metody Adamsa rzędu 4 ze stałym krokiem oraz Rungego-Kutty rzędu 4 ze zmiennym krokiem.

Każde z zadań badało układ dwóch równań różniczkowych:
\[
\begin{cases}
x_1' = x_2 + x_1 (0,9 - x_1^2 - x_2^2) \\
x_2' = -x_1 + x_2 (0,9 - x_1^2 - x_2^2)
\end{cases}
\]
na przedziale czasowym \( \langle0; 20\rangle \) dla czterech warunków początkowych:
\begin{center}
a) \( \begin{bmatrix} 10 \\ 9 \end{bmatrix} \),
b) \( \begin{bmatrix} 0 \\ 7 \end{bmatrix} \),
c) \( \begin{bmatrix} 7 \\ 0 \end{bmatrix} \),
d) \( \begin{bmatrix} 0,001 \\ 0,001 \end{bmatrix} \).
\end{center}

\section{Zadanie \textrm{I}. Metoda RK4 ze stałym krokiem}
Metoda Rungego-Kutty określona jest ogólnie wzorem:
\[ y_n = y_{n-1} + h \cdot \sum_{i=1}^m w_i k_i \text{,} \]
gdzie 
\(
k_i =
\begin{cases}
f(x_n, y_{n-1}), \quad \text{dla } i=1 \\
f(x_n + h\cdot\sum_{j=1}^{i-1} a_{ij}, y_{n-1} + h\cdot\sum_{j=1}^{i-1} a_{ij} k_j), \quad \text{dla } i=2,3,...,m
\end{cases}
\)

Jak wynika z przytoczonego równania, omawiana metoda należy do grupy metod jednokrokowych, tj.~do obliczenia kolejnego punktu rozwiązania wymagana jest znajomość jedynie punktu bezpośrednio go poprzedzającego.

Badana wersja metody, zwana również klasyczną, jest popularna w praktycznym zastosowaniu, ponieważ ma najlepszy stosunek dokładności (rząd \( p=4 \)) do liczby obliczeń (\( m=4 \) obliczenia wartości funkcji w każdym kroku).

\subsection{Szacowanie błędu}

W implementacji błąd metody szacowany jest wedle zasady podwójnego kroku. Wymaga to w~każdym kroku równoległego obliczenia kolejnego punktu zdyskretyzowanego rozwiązania poprzez wykonanie dwóch kroków dwukrotnie mniejszych. Następnie ,,konkurujące'' wartości należy podstawić do wzoru:
\[ \delta_n(h) = \frac{2^p}{2^p - 1} (y_n^{(2)} - y_n^{(1)}) \text{,}\]
gdzie \( y_n^{(2)} \) jest wartością obliczoną dwoma półkrokami, a \( y_n^{(1)} \) jednym zwykłym.

W podręczniku do przedmiotu zawarta była porada, by wykorzystywać otrzymany w metodzie podwójnego kroku dokładniejszy punkt \( y_n^{(2)} \) do dalszych obliczeń (przy jednoczesnym użyciu alternatywnego wzoru na szacowanie mniejszego błędu), zignorowałem ją jednak, by porównanie implementacji różnych metod różnicowych było wierniejsze.

\subsection{Graficzna analiza rozwiązania}
Jak zostało to zalecone w poleceniu, w celu znalezienia optimum wykonałem dla każdego punktu startowego kilka(naście) prób z~różnymi długościami kroku. Zaczynałem od znalezienia maksymalnej wartości kroku, dla której metoda pozostawała zbieżna. Przykładowo dla przebiegu przedstawionego na wykresie \ref{fig:1a} była to wartość ok.~0,01327. Co ciekawe, rozwiązanie było najbardziej skore rozbiec się na początku trajektorii (np.~dla wykresu \ref{fig:1a} -- prawy górny róg). Podobną tendencję widać dla wszystkich punktów startowych znajdujących się na zewnątrz pierścienia, w którym trajektoria ruchu ulega stabilizacji. Pierścień ten ma środek w okolicach punktu (0, 0), a jego promień mierzy niecałą 1 jednostkę. W jego otoczeniu nawet trajektorie oznaczone na wykresach jako te ze zbyt dużym krokiem (i początkowo wykazujące chęć ,,ucieczki'') nie przejawiały sporych odstępstw od domniemanej poprawnej wartości (potwierdzanej matlabową funkcją \texttt{ode45}). Dlatego też przy eksperymentalnym wyznaczaniu optymalnego kroku zwracałem uwagę właśnie na ,,ogonek'' wykresu. Zmiany dokładności wraz z krokiem były na nim najlepiej widoczne.

\begin{figure}[!htbp]
\centering
\includegraphics[width=14cm]{1a}
\centering
\caption{Porównanie wyników implementacji metody RK4 dla optymalnej i zbyt dużej długości kroku z~wbudowaną funkcją \texttt{ode45} -- warunki początkowe a)}
\label{fig:1a}
\end{figure}
\FloatBarrier

\begin{figure}[!htbp]
\centering
\includegraphics[width=14cm]{1b}
\centering
\caption{Porównanie wyników implementacji metody RK4 dla optymalnej i zbyt dużej długości kroku z~wbudowaną funkcją \texttt{ode45} -- warunki początkowe b)}
\label{fig:1b}
\end{figure}
\FloatBarrier

\begin{figure}[!htbp]
\centering
\includegraphics[width=14cm]{1c}
\centering
\caption{Porównanie wyników implementacji metody RK4 dla optymalnej i zbyt dużej długości kroku z~wbudowaną funkcją \texttt{ode45} -- warunki początkowe c)}
\label{fig:1c}
\end{figure}
\FloatBarrier

Warunki początkowe d) były na tyle korzystne, że możliwe było zastosowanie kroku o długości nawet ponad 2 jednostek czasowych. Dzięki takiemu szerokiemu zakresowi zbieżności udało mi się przedstawić na wykresie \ref{fig:1d} rzeczywiście ,,kanciastą'' trajektorię dla kroku zbyt dużego.

\begin{figure}[!htbp]
\centering
\includegraphics[width=14cm]{1d}
\centering
\caption{Porównanie wyników implementacji metody RK4 dla optymalnej i zbyt dużej długości kroku z~wbudowaną funkcją \texttt{ode45} -- warunki początkowe d)}
\label{fig:1d}
\end{figure}
\FloatBarrier

\section{Zadanie \textrm{II}. Metoda APC4 ze stałym krokiem}
Metody Adamsa wynikają z zamiany równania różniczkowego na równoważne mu równanie całkowe (na przedziale określonym długością kroku). Metody te należą do grupy metod wielokrokowych, tj.~obliczają następny punkt rozwiązania na podstawie kilku kolejnych punktów poprzedzających.

Metody wielokrokowe dzielą się na jawne (gdzie kolejne wartości \( y_n \) zależą jedynie od już obliczonych punktów) oraz niejawne (gdzie zależnością może być również wartość bieżąca).

Metody typu predyktor-korektor są praktycznymi implementacjami metod wielokrokowych, czyli rozwiązaniem hybrydowym łączącym zalety metod jawnych (ograniczenie ilości obliczeń) i niejawnych (zwiększenie dokładności i stabilności). Omawiana wersja algorytmu składa się w każdej iteracji z~następujących etapów:
\begin{enumerate}
    \item Predykcja: \( y_n^{[0]} = y_{n-1} + h \sum_{j=1}^k\beta_j f_{n-j} \),
    \item Ewaluacja: \( f_n^{[0]} = f(x_n, y_n^{[0]}) \)
    \item Korekcja: \( y_n = y_{n-1} + h \sum_{j=1}^k\beta_j^* f_{n-j} + h \beta_0^* f_n^{[0]} \)
    \item Ewaluacja: \( f_n = f(x_n, y_n) \)
\end{enumerate}

Rolą predyktora jest wyznaczenie dobrego punktu startowego dla korektora, który iteracyjnie rozwiązuje układ równań nieliniowych (wynikający z metody niejawnej Adamsa).

Implementowana metoda jest w wersji czterokrokowej, a więc w każdej iteracji potrzebuje 4~poprzednich punktów rozwiązania. Na start konieczne było zatem dostarczenie (wyłączając warunek początkowy) 3~kolejnych punktów wyliczonych z użyciem metody RK4 z poprzedniego zadania.

Jak poprzednio, krok jest stały na całym badanym zakresie, więc musiał zostać dostosowany do fragmentów największej zmienności.

\subsection{Szacowanie błędu}
Schemat szacowania błędu w metodzie predyktor-korektor Adamsa bazuje na różnicy między rozwiązaniem przybliżonym w kroku predyktora a tym po korekcie. W implementowanej metodzie rząd predyktora i korektora jest równy, można zatem zastosować wzór postaci:
\[ \delta_n(h) = \frac{c_{p+1}^*}{c_{p+1} - c_{p+1}^*} (y_n^{[0]} - y_n) \text{.}\]

\subsection{Graficzna analiza rozwiązania}
Biorąc pod uwagę wcześniejsze doświadczenia z metodą RK4 ze stałym krokiem, która dla zbyt dużych wartości \( h \) rozbiegała się już w pierwszych iteracjach, łatwo było mi wyznaczyć górny limit sensownych wartości kroku. Predyktor-korektor Adamsa miał jednak większe problemu z~badanym układem i warunkami początkowymi, dlatego uzyskanie przystępnych wyników wizualnych wymagało kilkukrotnego zmniejszenia długości kroku. Dla zbyt dużych kroków trajektorie często ,,przestrzeliwały'' miejsce oscylacji, a po zbiegnięciu do niego zachowywały pewien odstęp (tj.~miały promień mniejszy lub większy niż docelowy pierścień).

\begin{figure}[!htbp]
\centering
\includegraphics[width=14cm]{2a}
\centering
\caption{Porównanie wyników implementacji metody predyktor-korektor Adamsa 4 dla optymalnej i zbyt dużej długości kroku z~wbudowaną funkcją \texttt{ode45} -- warunki początkowe a)}
\label{fig:2a}
\end{figure}
\FloatBarrier

\begin{figure}[!htbp]
\centering
\includegraphics[width=14cm]{2b}
\centering
\caption{Porównanie wyników implementacji metody predyktor-korektor Adamsa 4 dla optymalnej i zbyt dużej długości kroku z~wbudowaną funkcją \texttt{ode45} -- warunki początkowe b)}
\label{fig:2b}
\end{figure}
\FloatBarrier

\begin{figure}[!htbp]
\centering
\includegraphics[width=14cm]{2c}
\centering
\caption{Porównanie wyników implementacji metody predyktor-korektor Adamsa 4 dla optymalnej i zbyt dużej długości kroku z~wbudowaną funkcją \texttt{ode45} -- warunki początkowe c)}
\label{fig:2c}
\end{figure}
\FloatBarrier

Ciekawy przypadek stanowi przykład ostatni (wykres \ref{fig:2d}), gdzie przestrzeń \( \langle-1; 1\rangle^2 \) widoczna jest w sporym powiększeniu. Pozwala to na zauważenie wspomnianej różnicy w promieniu tworzonych przez trajektorię okręgów (przy odpowiednio dużym kroku).

\begin{figure}[!htbp]
\centering
\includegraphics[width=14cm]{2d}
\centering
\caption{Porównanie wyników implementacji metody predyktor-korektor Adamsa 4 dla optymalnej i zbyt dużej długości kroku z~wbudowaną funkcją \texttt{ode45} -- warunki początkowe d)}
\label{fig:2d}
\end{figure}
\FloatBarrier

\section{Zadanie \textrm{III}. Metoda RK4 ze zmiennym krokiem}
Ostatnie zadanie wykorzystuje ponownie metodę Rungego-Kutty, ale używa prostego algorytmu do wyznaczenia długości kroku w każdej iteracji (zależnie od zmienności trajektorii). Pozwala to na oszczędność obliczeniową w miejscach, które na to pozwalają -- w przypadku kroku stałego konieczne było przyjęcie dla całego badanego zakresu wartości odpowiadającej przedziałowi największej zmienności.

Algorytm w każdej iteracji dokonuje ewaluacji współczynnika \( \alpha \), który decyduje o tym, czy krok powinien być zwiększony, czy zmniejszony. Nadmierne zmiany ograniczane są przez następujące parametry: parametr bezpieczeństwa \( s \), maksymalny stosunek kroku nowego do poprzedniego \( \beta \) oraz minimalną długość kroku \( h_{min} \).

\subsection{Szacowanie błędu}
Błąd jest szacowany identycznie, jak w zadaniu pierwszym -- program wykorzystuje zasadę podwójnego kroku. Uzyskana wartość używana jest w algorytmie korygującym długość kroku:
\[ \alpha = \min (\frac{|\mathbf{y_n}|\cdot \varepsilon_w + \varepsilon_b}{\mathbf{\delta_n}(h)})^{\frac{1}{p+1}} \text{,}\]
gdzie \( \varepsilon_w \) i \( \varepsilon_b \) to odpowiednio maksymalny błąd względny i bezwzględny zdefiniowany przez użytkownika. Wytłuszczone symbole \( \mathbf{y_n} \) i \( \mathbf{\delta_n} \) to wartości wektorowe, stąd współczynnik korekcji kroku \( \alpha \) obliczany jest dla elementu rozwiązania z największym błędem.

\subsection{Podsumowanie graficzne}
Wykresy nie pozostawiają wątpliwości: dynamiczne dostosowywanie kroku to świetne rozwiązanie, dzięki zastosowaniu którego możliwe jest uzyskanie ,,gładkiej'' trajektorii ruchu bez ręcznego eksperymentowania.

\begin{figure}[!htbp]
\centering
\includegraphics[width=14cm]{3a}
\centering
\caption{Porównanie wyników implementacji metody RK4 ze zmiennym krokiem z~wbudowaną funkcją \texttt{ode45} -- warunki początkowe a)}
\label{fig:3a}
\end{figure}
\FloatBarrier

\begin{figure}[!htbp]
\centering
\includegraphics[width=14cm]{3b}
\centering
\caption{Porównanie wyników implementacji metody RK4 ze zmiennym krokiem z~wbudowaną funkcją \texttt{ode45} -- warunki początkowe b)}
\label{fig:3b}
\end{figure}
\FloatBarrier

\begin{figure}[!htbp]
\centering
\includegraphics[width=14cm]{3c}
\centering
\caption{Porównanie wyników implementacji metody RK4 ze zmiennym krokiem z~wbudowaną funkcją \texttt{ode45} -- warunki początkowe c)}
\label{fig:3c}
\end{figure}
\FloatBarrier

\begin{figure}[!htbp]
\centering
\includegraphics[width=14cm]{3d}
\centering
\caption{Porównanie wyników implementacji metody RK4 ze zmiennym krokiem z~wbudowaną funkcją \texttt{ode45} -- warunki początkowe d)}
\label{fig:3d}
\end{figure}
\FloatBarrier

\section{Wnioski, porównanie}
Dla iloczynu kartezjańskiego badanych metod i danych warunków początkowych zebrałem dane analityczne podsumowujące ich efektywność. Przedstawiłem je w tabeli \ref{table:summary}.
\begin{table}[!htbp]
\centering
\begin{longtable}{|cc|c|c|c|c|c|}
\hline
    \multicolumn{2}{|c|}{\textbf{Metoda}} & \textbf{Warunki} & \textbf{Liczba} & \textbf{Czas} & \textbf{Średni} & \textbf{Maksymalny} \\
     & \setcounter{table}{0} & \textbf{początkowe} & \textbf{iteracji} & \textbf{obliczeń [s]} & \textbf{błąd} & \textbf{błąd} \\
    \hline\hline
    \multirow{4}{*}{\rotatebox[origin=c]{90}{RK4}} & \multirow{4}{*}{\rotatebox[origin=c]{90}{k. stały}} & a & 2667 & 0,1441 & 9,3044 \(\cdot10^{-4} \) & 2,4815 \(\cdot10^{\hphantom{-}0} \) \\ \cline{3-7}
     & & b & 1143 & 0,0624 & 1,9159 \(\cdot10^{-4} \) & 0,2186 \(\cdot10^{\hphantom{-}0} \) \\ \cline{3-7}
     & & c & 1143 & 0,0411 & 1,9159 \(\cdot10^{-4} \) & 0,2186 \(\cdot10^{\hphantom{-}0} \) \\ \cline{3-7}
     & & d &  201 & 0,0112 & 2,4078 \(\cdot10^{-7} \) & 3,7536 \(\cdot10^{-7} \) \\ \hline
    \hline
    \multirow{4}{*}{\rotatebox[origin=c]{90}{APC4}} & \multirow{4}{*}{\rotatebox[origin=c]{90}{k. stały}} & a & 20001 & 0,2426 & 1,0817 \(\cdot10^{-5} \) & 0,0859 \(\cdot10^{\hphantom{-}0}\) \\ \cline{3-7}
     & & b & 20001 & 0,1909 & 1,3332 \(\cdot10^{-6} \) & 0,0034 \(\cdot10^{\hphantom{-}0} \) \\ \cline{3-7}
     & & c & 20001 & 0,2067 & 1,3332 \(\cdot10^{-6} \) & 0,0034 \(\cdot10^{\hphantom{-}0} \) \\ \cline{3-7}
     & & d & 20001 & 0,1829 & 4,8791 \(\cdot10^{-8} \) & 7,7249 \(\cdot10^{-8} \) \\ \hline
    \hline
    \multirow{4}{*}{\rotatebox[origin=c]{90}{RK4}} & \multirow{4}{*}{\rotatebox[origin=c]{90}{k. zmienny}} & a & 77 & 0,0559 & 1,0684 \(\cdot10^{-4} \) & 6,4706 \(\cdot10^{-4} \) \\ \cline{3-7}
     & & b & 234 & 0,0192 & 3,3787 \(\cdot10^{-7} \) & 6,9776 \(\cdot10^{-7} \) \\ \cline{3-7}
     & & c & 133 & 0,0224 & 4,7237 \(\cdot10^{-6} \) & 1,1166 \(\cdot10^{-5} \) \\ \cline{3-7}
     & & d & 172 & 0,0201 & 3,7242 \(\cdot10^{-7} \) & 6,0967 \(\cdot10^{-7} \) \\ \hline
\end{longtable}
\caption{Zabrane dane analityczne}
\label{table:summary}
\end{table}
\FloatBarrier

Na podstawie zgromadzonych informacji widać od razu przewagę metody RK4 ze zmiennym krokiem nad pozostałymi. Realizuje ona swoje zadanie szybko, zamyka się w relatywnie niewielkiej liczbie iteracji, a błąd nie odstaje od konkurencyjnych wyników (przy czym istnieje możliwość bezpośredniego sterowania nim poprzez parametry algorytmu).

Jeśli chodzi o liczbę iteracji koniecznych do wykonania, by efekt wizualny był zadowalający, najgorzej poradził sobie predyktor-korektor Adamsa. W związku z tym obliczenia z jego użyciem były również najkosztowniejsze czasowo.

Moją uwagę zwrócił spory maksymalny błąd w metodach ze stałym krokiem dla warunków początkowych a) -- c). Pozwoliłem sobie więc na wykonanie dodatkowych badań i przedstawienie ich w formie wykresów.

Jak widać na rysunkach \ref{fig:a_err}, \ref{fig:b_err} i \ref{fig:c_err}, wspomniany błąd maksymalny występuje zaraz na początku obliczeń. Jest to zapewne związane z omawianym już początkowym dążeniem układu do rozbieżności. \\
Algorytm ze zmiennym krokiem zachowuje się tu zgoła odmiennie, zaczynając obliczenia z błędem \textit{minimalnym}. Niewątpliwie zdolności adaptacyjne zapewniają mu łagodny start. Potem algorytm próbuje dostosować się do wyznaczonego poziomu błędu maksymalnego, stabilizując się lub oscylując tuż pod nim.

Na przestrzeni wykresów utrzymuje się następująca tendencja: po stabilizacji najmniejszym błędem cechuje się algorytm Rungego-Kutty ze stałym krokiem, parę rzędów wielkości większy błąd ma algorytm predyktor-korektor i wreszcie największy błąd reprezentuje algorytm RK ze zmiennym krokiem. Inne wyniki przedstawia dopiero przypadek d) (rys. \ref{fig:d_err}), gdzie wszystkie metody wykazują błędy zbliżone, a największą dokładność przejawia zespół metod Adamsa. \\
Warto jednak pamiętać, że wartości błędów są w zadaniach jedynie szacowane -- nie da się ich wyznaczyć dokładnie, stąd szukanie stałego kroku optymalnego czy ocena jakości trajektorii przy kroku zmiennym dokonywane były głównie organoleptycznie.

\begin{figure}[!htbp]
\centering
\includegraphics[width=14cm]{a_err}
\centering
\caption{Porównanie normy błędu badanych implementacji dla warunków początkowych a)}
\label{fig:a_err}
\end{figure}
\FloatBarrier

\begin{figure}[!htbp]
\centering
\includegraphics[width=14cm]{b_err}
\centering
\caption{Porównanie normy błędu badanych implementacji dla warunków początkowych b)}
\label{fig:b_err}
\end{figure}
\FloatBarrier

\begin{figure}[!htbp]
\centering
\includegraphics[width=14cm]{c_err}
\centering
\caption{Porównanie normy błędu badanych implementacji dla warunków początkowych c)}
\label{fig:c_err}
\end{figure}
\FloatBarrier

\begin{figure}[!htbp]
\centering
\includegraphics[width=14cm]{d_err}
\centering
\caption{Porównanie normy błędu badanych implementacji dla warunków początkowych d)}
\label{fig:d_err}
\end{figure}
\FloatBarrier

\clearpage
\section{Wydruk wykorzystanego kodu}

\subsection{RK4conststep.m}
Implementacja metody Rungego-Kutty rzędu 4 ze stałym krokiem.
\begin{minted}{matlab}
function [t, result, errors] = RK4conststep(eqSys, tSpan, x0, stepSize)
    assert(length(tSpan) == 2, "TimeSpan must consist only of beginning and ending value");
    t = tSpan(1):stepSize:(tSpan(2) - eps);
    result = zeros(length(t), length(x0));
    result(1, :) = x0;
    errors = zeros(size(result));
    errors(1, :) = ones(1, length(x0)) * eps;
    errorConst = 16 / 15;

    for i=2:length(t)
        result(i, :) = RK4step(eqSys, result(i - 1, :), stepSize);
        % approximation error estimate
        helperResult = RK4step(eqSys, result(i - 1, :), stepSize / 2);
        helperResult = RK4step(eqSys, helperResult, stepSize / 2);
        errors(i, :) = abs(errorConst * (helperResult - result(i, :)));
        if any([isnan(errors(i, :)), isnan(result(i, :))])
            error("Method diverges.")
        end
    end
end
\end{minted}

\subsection{APC4conststep.m}
Implementacja metody predyktor-korektor Adamsa rzędu 4 ze stałym krokiem.
\begin{minted}{matlab}
function [t, result, errors] = APC4conststep(eqSys, tSpan, x0, stepSize)
    assert(length(tSpan) == 2, "TimeSpan must consist only of beginning and ending value");
    t = tSpan(1):stepSize:(tSpan(2) - eps);
    result = zeros(length(t), length(x0));
    result(1, :) = x0;
    evalResult = zeros(length(t), length(x0));
    evalResult(1, :) = eqSys(result(1, :));
    errors = zeros(size(result));
    errors(1, :) = ones(1, length(x0)) * eps;

    % Adams Predictor-Corrector (4-step)
    explicitBeta = [55 / 24, -59 / 24, 37 / 24, -9 / 24];
    implicitBeta = [251 / 720, 646 / 720, -264 / 720, 106 / 720, -19 / 720];
    % c*_{p+1} / (c_{p+1} - c*_{p+1}) = (19 / 720) / ((-251 / 720) - (19 / 720))
    errorConst = -19 / 270;

    % Adams PC needs k starting points, k = 4
    [~, RK4result, RK4errors] = RK4conststep(eqSys, [t(1), t(4) + stepSize / 2], x0, stepSize);
    result(2:4, :) = RK4result(2:4, :);
    errors(2:4, :) = RK4errors(2:4, :);
    evalResult(2, :) = eqSys(result(2, :));
    evalResult(3, :) = eqSys(result(3, :));
    evalResult(4, :) = eqSys(result(4, :));

    for i=5:length(t)
        % predictor (P_4)
        prediction = result(i - 1, :) + stepSize * (explicitBeta * evalResult((i - 4):(i - 1), :));
        % evaluator (E)
        evalResult(i, :) = eqSys(prediction);
        % corrector (C_4)
        result(i, :) = result(i - 1, :) + stepSize * (implicitBeta * evalResult((i - 4):i, :));
        % evaluator (E)
        evalResult(i, :) = eqSys(result(i, :));
        % approximation error estimate
        errors(i, :) = abs(errorConst * (prediction - result(i, :)));
        if any([isnan(errors(i, :)), isnan(result(i, :))])
            error("Method diverges.")
        end
    end
end
\end{minted}

\subsection{RK4varstep.m}
Implementacja metody Rungego-Kutty rzędu 4 ze zmiennym krokiem.
\begin{minted}{matlab}
function [t, result, errors] = RK4varstep(eqSys, tSpan, x0, initStepSize, minStepSize, absTolerance, relTolerance)
    assert(length(absTolerance) == 1, "Absolute tolerance must be a scalar value");
    assert(absTolerance > 0, "Absolute tolerance must be positive");
    assert(length(relTolerance) == 1, "Relative tolerance must be a scalar value");
    assert(relTolerance > 0, "Relative tolerance must be positive");
    assert(length(tSpan) == 2, "TimeSpan must consist only of beginning and ending value");
    t(1) = tSpan(1);
    result(1, :) = x0;
    errors(1, :) = ones(1, length(x0)) * eps;
    errorConst = 16 / 15;
    safetyCoeff = 0.9;
    maxStepMultiplier = 5;

    stepSize = initStepSize;
    i = 2;
    while t(i - 1) < tSpan(2)
        t(i) = t(i - 1) + stepSize;
        result(i, :) = RK4step(eqSys, result(i - 1, :), stepSize);
        % approximation error estimate
        helperResult = RK4step(eqSys, result(i - 1, :), stepSize / 2);
        helperResult = RK4step(eqSys, helperResult, stepSize / 2);
        errors(i, :) = abs(errorConst * (helperResult - result(i, :)));
        % step size correction
        tolerance = abs(result(i, :)) * relTolerance + absTolerance;
        alpha = min((tolerance ./ errors(i, :)) .^ (1 / 5));
        if isnan(alpha)
            alpha = 0.99;
        end
        if safetyCoeff * alpha >= 1
            stepSize = min([safetyCoeff * alpha * stepSize, maxStepMultiplier * stepSize, tSpan(2) - t(i)]);
            i = i + 1;
        else
            stepSize = safetyCoeff * alpha * stepSize;
            assert(stepSize > minStepSize, "Step below the specified limit");
        end
    end
end
\end{minted}

\subsection{RK4step.m}
Pojedynczy krok klasycznej (rzędu 4) metody Rungego-Kutty.
\begin{minted}{matlab}
function [result] = RK4step(eqSys, x, stepSize)
    % classic Runge-Kutta method
    w = [1, 2, 2, 1];
    k = zeros(4, length(x));
    k(1, :) = eqSys(x);
    k(2, :) = eqSys(x + stepSize * k(1, :) / 2);
    k(3, :) = eqSys(x + stepSize * k(2, :) / 2);
    k(4, :) = eqSys(x + stepSize * k(3, :));
    result = x + stepSize * (w * k) / 6;
end
\end{minted}

\subsection{equationsystem.m}
Dane wejściowe -- układ równań różniczkowych.
\begin{minted}{matlab}
function [result] = equationsystem(x)
    assert(length(x) == 2, "x must be of length 2");
    result = [
        x(2) + x(1) * (0.9 - x(1)^2 - x(2)^2);  % x1'
        -x(1) + x(2) * (0.9 - x(1)^2 - x(2)^2)  % x2'
    ];
end
\end{minted}

\subsection{getinitialposition.m}
Dane wejściowe -- punkty startowe.
\begin{minted}{matlab}
function [initialPos] = getinitialposition(example)
    if example == "a"
        initialPos = [10; 9];
    elseif example == "b"
        initialPos = [0; 7];
    elseif example == "c"
        initialPos = [7; 0];
    elseif example == "d"
        initialPos = [0.001; 0.001];
    else
        error("Example must be one of: a, b, c, d");
    end
end
\end{minted}

\subsection{getrange.m}
Dane wejściowe -- badany przedział czasowy.
\begin{minted}{matlab}
function [range] = getrange()
    range = [0, 20];
end
\end{minted}

\subsection{project4.m}
Zestaw funkcji do rysowania i opisu testowanych metod.
\begin{minted}{matlab}
function [] = project4(taskNo)
    if taskNo == 1
        task1();
    elseif taskNo == 2
        task2();
    elseif taskNo == 3
        task3();
    else
        error("TaskNo must be one of: 1, 2, 3");
    end
end

function [] = task1()
    % Absolute maximum:
    % * 0.01327215
    % * 0.049965
    % * 0.049965
    % * 2.225

    conststepwrapper(1, "a", 0.0075, 0.01327, ["northwest", "northeast"]);
    conststepwrapper(1, "b", 0.0175, 0.04996, ["northeast", "northeast"]);
    conststepwrapper(1, "c", 0.0175, 0.04996, ["northeast", "northeast"]);
    conststepwrapper(1, "d", 0.1, 0.5, ["northeast", "southeast"]);
end

function [] = task2()
    % Absolute maximum:
    % * 0.006790043245
    % * 0.0255
    % * 0.0255
    % * 2.488445; watch out: 0.19647 - 0.74075

    conststepwrapper(2, "a", 0.001, 0.004, ["northwest", "northeast"]);
    conststepwrapper(2, "b", 0.001, 0.01, ["northeast", "northeast"]);
    conststepwrapper(2, "c", 0.001, 0.01, ["northeast", "northeast"]);
    conststepwrapper(2, "d", 0.001, 0.05, ["northeast", "southeast"]);
end

function [] = task3()
    % OK parameters:
    % * 0.01, 10e-15, 10e-5, 10e-5
    % * 0.1, 10e-15, 10e-10, 10e-7
    % * 0.1, 10e-15, 10e-7, 10e-6
    % * 0.1, 10e-15, 10e-8, 10e-7

    varstepwrapper("a", 0.01, 10e-15, 10e-5, 10e-5, ["northwest", "southeast"]);
    varstepwrapper("b", 0.1, 10e-15, 10e-10, 10e-7, ["northeast", "southeast"]);
    varstepwrapper("c", 0.1, 10e-15, 10e-7, 10e-6, ["northeast", "southeast"]);
    varstepwrapper("d", 0.1, 10e-15, 10e-8, 10e-7, ["northeast", "southeast"]);
end

function [] = conststepwrapper(taskNo, example, optimalStep, tooBigStep, legendLocation)
    assert(length(legendLocation) == 2, ...
        "LegendLocation must contain two strings: one for graph and one for error comparison");
    if taskNo == 1
        method = @RK4conststep;
        methodName = 'RK4';
    elseif taskNo == 2
        method = @APC4conststep;
        methodName = 'APC4';
    else
        error("TaskNo must be one of: 1, 2");
    end
    tic;
    [tChosen, yChosen, errChosen] = method(@equationsystem, getrange(), getinitialposition(example), optimalStep);
    timeChosen = toc;
    [~, yBigger, ~] = method(@equationsystem, getrange(), getinitialposition(example), tooBigStep);
    [~, yOde] = ode45(@(t, x) equationsystem(x), getrange(), getinitialposition(example));
    % comparison of step size
    figure("Name", ['Graph comparison for ', num2str(taskNo), char(example)]);
    colororder(["#000000", "#e6194b", "#3cb44b"]);
    plot(yChosen(:, 1), yChosen(:, 2), "LineWidth", 1);
    hold on;
    plot(yBigger(:, 1), yBigger(:, 2));
    plot(yOde(:, 1), yOde(:, 2));
    hold off;
    legend([methodName, ', step: ', num2str(optimalStep)], ...
        [methodName, ', step: ', num2str(tooBigStep), ' (too big)'], ...
        "ODE45", "Location", legendLocation(1));
    xlabel("x1");
    ylabel("x2");
    % summary
    ['Summary for ', num2str(taskNo), char(example)]
    iterationCount = length(tChosen)
    calculationTime = timeChosen
    averageError = mean(cellfun(@norm, num2cell(errChosen, 2)))
    maximalError = max(cellfun(@norm, num2cell(errChosen, 2)))
end

function [] = varstepwrapper(example, initStepSize, minStepSize, absTolerance, relTolerance, legendLocation)
    tic;
    [tChosen, yChosen, errChosen] = RK4varstep(@equationsystem, getrange(), ...
        getinitialposition(example), initStepSize, minStepSize, absTolerance, relTolerance);
    timeChosen = toc;
    [~, yOde] = ode45(@(t, x) equationsystem(x), getrange(), getinitialposition(example));
    % comparison of step size
    figure("Name", ['Graph comparison for 3', char(example)]);
    colororder(["#000000", "#e6194b"]);
    plot(yChosen(:, 1), yChosen(:, 2), "LineWidth", 1);
    hold on;
    plot(yOde(:, 1), yOde(:, 2));
    hold off;
    legend("RK4, dynamic step", "ODE45", "Location", legendLocation(1));
    xlabel("x1");
    ylabel("x2");
    % summary
    ['Summary for 3', char(example)]
    iterationCount = length(tChosen)
    calculationTime = timeChosen
    averageError = mean(cellfun(@norm, num2cell(errChosen, 2)))
    maximalError = max(cellfun(@norm, num2cell(errChosen, 2)))
end
\end{minted}

\end{document}